\newtheorem{theorem}{Theorem}[section]
\newtheorem{lemma}{Lemma}[section]
\newtheorem{defi}{Definition}[section]
\newtheorem{corollary}{Corollary}[section]

\chapter{Lower Oracle Query Bounds}
\label{ch:oracle}

In this chapter we present the quantum oracle framework.  We first
define the quantum oracle model and oracle query complexity, then
present a general theorem due to Ambainis \cite{ambainis00quantum}.
In Section \ref{sec:lplqb} we derive a key lemma used throughout the
thesis and apply it to a simple problem.  We then show that for the
problem of determining the oracle string, Ambainis' Theorem can not
attain an asymptotically tight lower query bound in Sections
\ref{sec:dos} and \ref{sec:torf1}.

Our primary purpose is to demonstrate the power and flexibility of
Ambainis' Theorem in attaining lower bounds on broad classes of
Boolean functions and simplifying previous proofs.  We investigate
Boolean functions with partial symmetry in Section \ref{sec:lqbsym}.
We apply the result of that section to attain asymptotically tight
lower oracle query bounds of $\Omega(\sqrt{N})$, $\Omega(\sqrt{N})$,
$\Omega(N)$, and $\Omega(N)$ for the functions AND, OR, MAJORITY, and
PARITY respectively in Section
\ref{sec:AOMP}.  Finally we prove $\Omega(\sqrt{N})$ oracle
queries are required to compute any $N$-bit nonconstant symmetric
Boolean function in Section \ref{sec:lqbnsf}.

\section{Preliminaries}

\subsubsection{Useful Definitions}

We will prove lower bounds for functions from $\{0,1\}^{N}$ to
$\{0,1\}$, we will call such functions \emph{$N$-bit Boolean
functions}, or just \emph{Boolean functions}.  In our proofs we will
frequently need the notion of the \emph{Hamming weight} of a bit
string: this is the number of 1's in the string.  We denote the
Hamming weight of a bit string $x$ as $|x|$.  We also will frequently
refer to all inputs which differ from a particular input in only one
bit, we will call inputs which differ in a single bit \emph{Hamming
neighbors}.

\subsubsection{Quantum Oracle Models}
\label{sec:qom}

In the quantum oracle model, we have an oracle that holds an $N$-bit
input string.  Our task is to determine the value of some fixed
Boolean function of the oracle string, using as few oracle queries as
possible.  An oracle query is a question of the form: ``What is the
$i$th bit of the oracle string?''  The quantum oracle model is a
special case of Ambainis' more general quantum adversary model
\cite{ambainis00quantum}, which we describe below.

In the quantum adversary model, we run an algorithm against an oracle
that contains a superposition of inputs.  Let $S$ be a subset of the
possible inputs $\{0,1\}^{N}$.  Algorithms in the quantum adversary
model will work in the Hilbert space $H = H_{A} \otimes H_{I}$, where
$H_{A} = \Complex^{2^{m}}$ is the Hilbert space of our $m$-qubit
memory register, and $H_{I} = \Complex^{|S|}$ is the Hilbert space
spanned by basis vectors $|x\rangle$ corresponding to the elements of
$S$.  We think of $H_{A}$ as our algorithm space, and $H_{I}$ as our
input space.  The tensor product of two vector spaces $A$ and $B$,
denoted $A \otimes B$, is a new vector space spanned by all possible
pairs $(i,j)$ of basis vectors $i$ from the first space and $j$ from
the second space. Thus $H = \Complex^{|S|2^{m}}$.

We can represent the basis states of our algorithm space as $|i, b,
z\rangle$, where $i$ consists of $\lceil\log{N}\rceil$ bits, $b$ is a
single bit, $z$ denotes all other bits our quantum algorithm requires.
We define the oracle transformation $O$ as the unitary operator that
takes any eigenstate $|i,b,z\rangle \otimes |x\rangle$ to $|i, b
\oplus x_{i}, z\rangle \otimes |x\rangle$. The first 
$\lceil\log{N}\rceil$ bits of the subspace defined by a particular
input $|x\rangle$ is the index $i$ to the oracle bit $x_{i}$ that we
are querying. $O$ is a permutation matrix.

A quantum algorithm that performs $T$ queries is just a sequence of
unitary transformations
\[U_{0}\rightarrow O\rightarrow U_{1} \rightarrow O \ldots \rightarrow U_{T-1} \rightarrow O \rightarrow U_{T},\]
where $U_{i}$ is an arbitrary unitary transformation that does not
depend on the oracle, and $O$ is the oracle transformation.  (Recall
that unitary transformations are reversible and preserve the
normalization of the state vector.)

The standard oracle query model is just an instance of the quantum
adversary model where the input space is spanned by a single
eigenstate $|x\rangle$.  In this case a quantum algorithm starts with
the state $|0\rangle \otimes |x\rangle$, applies $U_{0}, O, U_{1},
\ldots, O, U_{T}$, and then measures the final state.  The rightmost
bit of the measured state of the algorithm space is the output of the
algorithm on $x$.  The algorithm computes $f$ in the bounded error
setting if for every input $x \in \{0,1\}^{N}$, the output is $f(x)$
with some constant probability.

The measure of complexity in both the quantum oracle model and quantum
adversary model is the number of oracle queries.  It should be noted
that querying the oracle is not always the most time consuming portion
of an algorithm.  For example, to factor an $N$ bit integer that the
oracle holds, we can determine the integer in $N$ queries.  However,
we must then do $2^{N^{\Omega(1)}}$ additional steps in the classical
case, or $N^{2}\log^{O(1)}{N}$ additional steps in the quantum case to
factor the number using the best known algorithms
\cite{beals98quantum}. Nonetheless we restrict our attention to the 
number of oracle queries required, as it is clearly a lower bound on
the overall running time of the algorithm.  Classical bounds for query
complexity are well studied, making comparisons between quantum and
classical oracle query complexity possible.

\subsubsection{A Lower Query Bound Proving Framework due to Ambainis} 

Ambainis \cite{ambainis00quantum} proved the following fundamental
result:

\begin{theorem}[Ambainis]
\label{th:amb}
Let $f$ be an $N$-bit Boolean function and let $X$ and $Y$ be two sets
of inputs such that $f(x) \neq f(y)$ whenever $x \in X$ and $y \in Y$.
Let $P \subseteq X \times Y$ be a relation between $X$ and $Y$ with
the following properties:

\begin{enumerate}

\item
Every element of $X$ is related to at least $m$ elements of $Y$ by
$P$.

\item 
Every element of $Y$ is related to at least $m'$ elements of $X$ by
$P$.

\item 
For all $i$, every element $x \in X$ is related to at most $l$
elements $y \in Y$ such that $x_{i} \neq y_{i}$

\item 
For all $i$, every element $y \in Y$ is related to at most $l'$
elements $x \in X$ such that $x_{i} \neq y_{i}$

\end{enumerate}

Then
\[\Omega\left(\sqrt{\frac{mm^{\prime}}{ll^{\prime}}}\right)\] 
oracle queries are required by any quantum algorithm to compute $f$ in
the bounded error setting.
\end{theorem}

This theorem will be the primary means of proving lower bounds in this
thesis.  Its appeal is twofold; first, it leads to reasonably simple
proofs, and second, it unifies many existing lower bounds into a
single framework.

\paragraph{The Quantum Adversary:}

Lower bounds for both classical and quantum algorithms have been
attained in the past by analyzing the behavior of the algorithm in the
face of an adversary.  Ambainis uses a \emph{quantum adversary} as the
basis for his lower bound theorem; instead of running a quantum
algorithm against one input, it is run against a superposition of
inputs.  The formulation of the quantum adversary and the
justification for why a lower bound in the quantum adversary model is
also a lower bound in the standard oracle query model below are
adapted and somewhat simplified from Ambainis' paper
\cite{ambainis00quantum}.

Ambainis observed that any standard oracle query algorithm $A$ that
operates only on a single input is also correct in the quantum
adversary model.  Consider a standard oracle query algorithm $A$ that
uses $T$ queries and succeeds with probability $p$.  We assume our
quantum memory register begins in the $\left| 0 \right\rangle$ state.
For every input $x \in \{0,1\}^{N}$, the algorithm transforms the
initial state $\left| 0 \right\rangle \otimes \left| x \right\rangle$
into the final state
\[
\left(
  \alpha_{x}\left|f(x)\right\rangle
  +
  \beta_{x}|\overline{f(x)}\rangle
\right)
\otimes
\left|w_{x}\right\rangle
\otimes
\left|x\right\rangle
\]
where $|\alpha_{x}|^{2} + |\beta_{x}|^{2} = 1$, and $|\alpha_{x}|^{2}
\ge p$ (so $|\beta_{x}|^{2} \le 1-p$, and finally $|w_{x}\rangle$ is 
the remaining state of the memory. The same algorithm will take any
superposed input state
\[|0\rangle \otimes \sum_{x \in S} \frac{1}{\sqrt{\left|S\right|}} 
\left| x \right\rangle\]
to the final state
\[\frac{1}{\sqrt{|S|}} \sum_{x \in S} 
\left(
  \alpha_{x}\left|f(x)\right\rangle
  +
  \beta_{x}|\overline{f(x)}\rangle
\right)
\otimes
\left|w_{x}\right\rangle
\otimes
\left|x\right\rangle.
\]
Note that in this end state, if the oracle is measured to be $x$, a
measurement of the memory register will yield $f(x)$ with probability
$|\alpha_{x}|^{2} \ge p$.  Ambainis places a lower bound on the number of
oracle queries required by any quantum adversary algorithm to produce
such an end state from such a start state.

This argument gives us a lower bound in the standard oracle query
model as well.  If $T$ queries are required of a quantum adversary
algorithm to establish a correct end state with probability $p$, then
no standard oracle query algorithm $A$ can compute $f$ with bounded
probability $p$ in $t < T$ queries. 

This quantum adversary method provides a framework for proving a great
diversity of lower bounds.  Indeed, nearly every lower bound in this
thesis is proved directly or indirectly using Theorem \ref{th:amb}.
As useful as the theorem is, it does have limitations, one of which is
discussed in Section \ref{sec:torf1}.

\section{A Lemma for Proving Lower Query Bounds}
\label{sec:lplqb}

The sensitivity of an input $x$ of a function $f$ is the number of
Hamming neighbors $y$ of $x$ such that $f(x) \neq f(y)$.  We will use
the following lemma to establish lower bounds for functions based
solely on their sensitivity for a particular input.

\begin{lemma}
\label{lm:1xky}
Let $f$ be an $N$-bit Boolean function.  If some input $x$ has $k$
Hamming neighbors $y$ such that $f(x) \neq f(y)$, then
$\Omega(\sqrt{k})$ oracle queries are required to compute $f$ in the
bounded error setting.
\end{lemma}

\begin{proof}
To prove Lemma \ref{lm:1xky} we will apply Theorem \ref{th:amb}.  Let
$X$ contain the single input $x$ from the statement of the lemma.  Let
$Y$ contain the $k$ Hamming neighbors of $x$ such that $f(x)
\neq f(y)$ when $y \in Y$.  Let $P$ be the complete relation from $X$
to $Y$.

Then $m = k$, since $xPy$ for each of the $k$ elements $y \in Y$.
Since no two Hamming neighbors of $x$ differ from $x$ in the same bit
position, we have $l = 1$.  Since there is only one element $x$ in $X$
and $xPy$ for each $y \in Y$, we have $m^{\prime} = l^{\prime} = 1$.
The result now follows immediately from Theorem \ref{th:amb}.
\end{proof}

This lemma will not in general provide us with the best lower bound
that can be attained from Ambainis' Theorem \ref{th:amb}: we are
maximizing $m/l$, but we make no attempt to maximize
$m^{\prime}/l^{\prime}$.  The strongest results of Ambainis' Theorem
frequently arise from maximizing both.  We will maximize both when
dealing with partially symmetric functions in Theorem \ref{th:ptsym}
and graph connectivity in Theorem \ref{th:gc}.  The \emph{singleton}
class of functions described in Sections \ref{sec:dos} and
\ref{sec:torf1} is a class for which Lemma \ref{lm:1xky} obtains 
optimal results.

\subsection{Application to Generalized XOR} 
\label{sec:gXOR}

To see how simple proofs can be with Lemma \ref{lm:1xky} we provide an
example.  Generalized XOR is the $N$-bit Boolean function that is 0 if
and only if its input bits are all the same.

\begin{theorem}
\label{th:gXOR}
$\Omega(\sqrt{N})$ oracle queries are required to compute
the generalized XOR of $N$ bits in the bounded error setting.
\end{theorem}

\begin{proof}
The generalized XOR of $0^{N}$ is $0$, and the generalized XOR of the
$N$ Hamming neighbors of $0^{N}$ is 1.  The theorem follows from Lemma
\ref{lm:1xky} with $k = N$.
\end{proof}

This lower bound is asymptotically tight: Beals et al.\ provide
$O(\sqrt{N})$ oracle query algorithms for computing the AND or OR of
$N$ bits in the bounded error setting \cite{beals98quantum}, and the
generalized XOR of $N$ bits is just $\overline{AND \vee
\overline{OR}}$.  Unfortunately, Ambainis' Theorem does not always 
perform this well, as we demonstrate in the next two sections.

\section{Determining the Oracle String}
\label{sec:dos}

Consider the problem of determining the oracle string.  We prove two
lower bounds for this problem, one through Lemma $\ref{lm:1xky}$, and
another by using a known lower bound for PARITY.

\begin{theorem}
\label{th:dosbad}
$\Omega(\sqrt{N})$ oracle queries are required to determine an $N$-bit
oracle string in the bounded error setting.
\end{theorem}

\begin{proof}
Let $f$ be the $N$-bit Boolean function that is 1 if and only if its
input is the same as the oracle string $x$.  For each Hamming neighbor
$y$ of $x$, we have $f(x) \neq f(y)$.  The result then follows from
Lemma \ref{lm:1xky}.
\end{proof}

In the classical case $N$ oracle queries are required to determine the
oracle string.  If the lower bound of $\Omega(\sqrt{N})$ oracle
queries were asymptotically tight, then there would be quadratic
improvement over the classical case for this problem.  This seems
plausible in light of Grover's algorithm.  However, $\Omega(N)$ lower
bounds are known for Boolean functions in the quantum oracle model, we
can thus infer that Theorem \ref{th:dosbad} is not asymptotically
tight.

\begin{theorem}
\label{th:dosgood}
$N/2$ oracle queries are required to determine an $N$-bit oracle
string in the bounded error setting.
\end{theorem}

\begin{proof}
Once we compute the oracle string, we can compute its PARITY with no
additional queries.  Beals et al.\ proved that $N/2$ oracle queries
are required to compute PARITY for an $N$-bit oracle string in the
bounded error setting \cite{beals98quantum}.
\end{proof}

We now see an apparent limitation of Ambainis' Theorem; the result
attained was quadratically worse than the asymptotically tight lower
bound.  We will show in the next section that the weak lower bound in
Theorem \ref{th:dosbad} is indeed the best that can be attained
through application of Ambainis' Theorem.

\section{Singleton Functions}
\label{sec:torf1}

An $N$-bit Boolean function that evaluates to 1 for exactly one input
is a singleton function.  We will show that for any $N$-bit singleton
function Ambainis' Theorem can always attain a lower query bound of
$\Omega(\sqrt{N})$, but no better.

The proof of the $\Omega(\sqrt{N})$ lower query bound for determining
the oracle string in Theorem \ref{th:dosbad} can equally well be
applied to any singleton function.

\begin{corollary}
\label{th:1sqn}
$\Omega(\sqrt{N})$ oracle queries are required to compute any $N$-bit
singleton function in the bounded error setting.
\end{corollary}

\begin{theorem}
\label{th:best}
$\Omega(\sqrt{N})$ is the best lower bound on the number of oracle
queries required to compute an $N$-bit singleton function that can be
attained from Theorem \ref{th:amb}.
\end{theorem}

\begin{proof}
We will consider every possible choice of $X$, $Y$, and $P$ in Theorem
\ref{th:amb}.

Without loss of generality, let $X$ contain the single input $x$ such
that $f(x) = 1$.  Let $Y$ be any subset of the $2^{N} - 1$ bit strings
for which $f(y) = 0$.  If $P$ is not the complete relation $X \times
Y$ then some element $y \in Y$ is not related to $x$ by $P$, so the
$m^{\prime}$ component of Theorem \ref{th:amb} is $0$, giving a
vacuous lower bound.  Thus $P$ is the complete relation.

Here $m = |Y|$, and $m' = l' = 1$.  The final parameter $l$ is the
maximum over all $i$ of the number of $y \in Y$ that differ from $x$
in the $i$th bit position.

From Ambainis' Theorem $\Omega(\sqrt{m/l})$ oracle queries are
required to compute $f$.  We will prove that $m/l \le N$.  Let $Y_{i}$
be the number of $y \in Y$ that differ from $x$ in the $i$th bit, so
that $l = \max_{i}\{Y_{i}\}$.  If we construct $Y$ by adding elements
to it one at a time, then for each element we increment $m$, and at
least one $Y_{i}$.  After the first element $m = l = 1$, and by the
pigeonhole principle $m/\max_{i}\{Y_{i}\} \le N$.
\end{proof}

This result coupled with the $N/2$ lower bound for the singleton
function of determining the oracle string leads immediately the the
following corollary.

\begin{corollary}
\label{cor:ambbad}
There exist functions for which Ambainis' Theorem \ref{th:amb} can not
attain asymptotically tight lower bounds.
\end{corollary}

Despite this limitation, the theorem can still prove lower bounds for
many Boolean functions.  From this point forward all our lower bounds
are directly or indirectly attained through Ambainis' Theorem.

\section{Partially Symmetric Functions} 
\label{sec:lqbsym}

\begin{defi}
\label{def:sym}
A symmetric function is a Boolean function whose value depends only on
the Hamming weight of the input.
\end{defi}

Recall that the Hamming weight of a bit string is the number of 1's it
has.

Consider the special class of $N$-bit Boolean functions $f$ such that
$f(x) = 1$ for all inputs $x$ of Hamming weight $a$, and $f(x) = 0$
for all inputs $x$ of Hamming weight $b$.  Without loss of generality
assume $a < b$.  We will attain a lower bound depending only on the
parameters $a$ and $b$.

\begin{theorem}
\label{th:ptsym}
Let $f$ be an $N$-bit Boolean function such that for some $a < b$,
$|x| = a$ implies $f(x) = 1$, and $|x| = b$ implies $f(x) = 0$.  Then
\[
\Omega\left(\sqrt{\frac{(N-a)b}{(b-a)^{2}}}\right)
\]
oracle queries are required to compute $f$ in the bounded error
setting.
\end{theorem}

\begin{proof}
To prove Theorem \ref{th:ptsym} we apply Theorem \ref{th:amb}.

Let $X$ be the strings of Hamming weight $a$, and let $Y$ be the
strings of Hamming weight $b$.  If we think of the bit strings in $X$
and $Y$ as defining subsets of $\{1,2,\ldots,N\}$ where a 1 in the
$i$th position means the set contains $i$, then $P$ is just $\subset$,
the proper subset relation.

Each set $x \in X$ is a subset of $m = {N-a \choose N-b}$ sets $y \in
Y$.  For any element $i \in x$, there are $l = {N-a-1 \choose N-b}$
supersets $y \in Y$ such that $i \in y$ and $x \subset y$.

Each set $y \in Y$ is a superset of $m' = {b \choose a}$ sets $x \in
X$.  For any element $i \in y$, there are $l' = {b-a \choose a}$
subsets $x \subset y$ such that $i \not\in x$.

Thus,
\[\frac{mm'}{ll'} = \frac{{N-a \choose N-a}{b \choose a}}
	                 {{N-a-1 \choose b-a-1}{b-1 \choose a}} 
                  = 
		     \frac{\left(N-a\right)b}{\left(b-a\right)^{2}},
\]
so by Theorem \ref{th:amb}
\[
\Omega\left(\sqrt{\frac{(N-a)b}{(b-a)^{2}}}\right)
\]
oracle queries are required to compute $f$.
\end{proof}

Like Lemma \ref{lm:1xky}, this result allows us to easily attain lower
bounds for many functions.  In general, the closer $a$ and $b$ are to
each other and to $N/2$ the better.  We illustrate the particular
success of Theorem \ref{th:ptsym} in the case of symmetric functions
in the following two sections, and with relatively less success in the
case of non-trivial monotone graph properties in Section
\ref{sec:lqbnmgp}.

\section{AND, OR, MAJORITY, and PARITY}
\label{sec:AOMP}

We can use Theorem \ref{th:ptsym} to easily prove lower bounds for the
well-known symmetric functions AND, OR, MAJORITY, and PARITY.  While
these results have been previously established by Beals et al.\
\cite{beals98quantum}, the ease with which they are proved through
Ambainis' Theorem is noteworthy.

\subsubsection{AND and OR}

AND is the $N$-bit Boolean function that evaluates to 1 if and only if
its input is $1^{N}$, and OR is the $N$-bit Boolean function that
evaluates to 0 if and only if its input is $0^{N}$.

\begin{theorem}
\label{th:AND}
$\Omega(\sqrt{N})$ oracle queries are required to compute
the AND or the OR of $N$ bits in the bounded error setting.
\end{theorem}

\begin{proof}
Let $a = N-1$ and $b = N$.  AND evaluates to 0 for all inputs of
Hamming weight $a$, and 1 for all inputs of Hamming weight $b$.
Therefore by Theorem \ref{th:ptsym}
\[\Omega\left(\sqrt{\frac{(N
- (N-1))N}{(N-(N-1))^{2}}}\right) =
\Omega\left(\sqrt{N}\right)\]
oracle queries are required to compute AND in the bounded error
setting.  A virtually identical proof with $a = 0$ and $b = 1$ follows
for OR.
\end{proof}

Beals et al.\ proved that these lower bounds are asymptotically tight
in the bounded error setting \cite{beals98quantum}.  Observe that we
could have just as easily used Theorem \ref{th:1sqn} to attain the
same lower bounds, as AND and OR are singleton functions.  In the
classical case exactly $N$ oracle queries are required to compute AND
or OR.  Thus, quadratic improvement is possible in the quantum bounded
error setting.  

\subsubsection{MAJORITY and PARITY}

It is tempting to hope that all Boolean functions, or at least all
symmetric Boolean functions, realize the quadratic speedup of AND and
OR.  Unfortunately the functions MAJORITY and PARITY show that for
some problems the speedup is at best a constant factor.

MAJORITY is the $N$-bit Boolean function that evaluates to 1 if and
only if more than half of the input bits are 1.  PARITY is the $N$-bit
Boolean function that evaluates to 1 if and only if the input has an
even Hamming weight.

\begin{theorem}
\label{th:MAJORITY}
$\Omega(N)$ oracle queries are required to compute the MAJORITY or the
PARITY of $N$ bits in the bounded error setting.
\end{theorem}

\begin{proof}
Let $a = \left\lfloor N/2 \right\rfloor$ and $b = a+1$.  MAJORITY
takes on a 0 for all inputs of Hamming weight $a$ and a 1 for all
inputs of Hamming weight $b$.  PARITY takes on one of $\{0,1\}$ for
all inputs of Hamming weight $a$ and the other for all inputs of
Hamming weight $b$.  Therefore by Theorem \ref{th:ptsym}
\[\Omega\left(\sqrt{\frac{\left(N - \left\lfloor\frac{N}{2}\right\rfloor\right)\left(\left\lfloor\frac{N}{2}\right\rfloor+1\right)}{\left(\left\lfloor\frac{N}{2}\right\rfloor+1-\left\lfloor\frac{N}{2}\right\rfloor\right)^{2}}}\right) =
\Omega\left(N\right) \]
oracle queries are required to compute either function.
\end{proof}

Since both classical and quantum algorithms can compute any function
of $N$ bits with $N$ oracle queries, this lower bound is
asymptotically tight.  Beals et al.\ previously proved these results
\cite{beals98quantum}; again, the simplicity that Ambainis' Theorem
affords is the main point of interest.

\section{Nonconstant Symmetric Functions}
\label{sec:lqbnsf}

AND, OR, MAJORITY, and PARITY are all symmetric functions, and the
lower bounds attained in the preceding section are all asymptotically
tight.  Our success leads us to investigate what we can prove about
symmetric functions in general.

\begin{theorem}
\label{th:sym}
$\Omega(\sqrt{N})$ oracle queries are required to compute any $N$-bit
nonconstant symmetric Boolean function in the bounded error setting.
\end{theorem}

\begin{proof}
For any $N$-bit nonconstant symmetric Boolean function, there are
Hamming weights $a$ and $b = a+1$ with $0 \le a \le N-1$ such that the
function differs on inputs of Hamming weights $a$ and $b$.  Otherwise
the function would be constant.

By Theorem \ref{th:ptsym}
\[\Omega\left(\sqrt{\frac{(N - a)(a+1)}{(a+1-a)^{2}}}\right) =
\Omega\left(\sqrt{\left(N-a\right)\left(a+1\right)}\right) = 
\Omega\left(\sqrt{N}\right)\]
oracle queries are required to compute the function.  
\end{proof}

Depending on the value of $a$ in Theorem \ref{th:sym} we may be able
to provide a better lower bound than $\Omega(\sqrt{N})$.  This result
was previously established by Beals et al.\ through a more complicated
method of polynomials \cite{beals98quantum}.  No better result can be
attained for the class of all symmetric Boolean functions as Beals et
al.\ provide $O(\sqrt{N})$ oracle query algorithms to compute the
symmetric functions AND and OR \cite{beals98quantum}.  Our success
here leads us to apply the results of Chapter \ref{ch:oracle} to graph
properties, which are very well studied in the classical oracle query
model, in Chapter \ref{ch:graph}.  While graph properties are not
necessarily ``symmetric'' in the sense of Definition \ref{def:sym},
they do display a kind of symmetry.

