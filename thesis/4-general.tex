\newtheorem{cond}{Condition}[section]
\newtheorem{algoi}{Algorithm}[section]

\chapter{Boolean Functions}
\label{ch:general}

In this chapter we consider increasingly general classes of functions.
We first examine lower bounds for tree functions in Section
\ref{sec:tree} and prove $\Omega(\sqrt[4]{N})$ oracle
queries are required to compute a tree function of $N$ bits.

We define nondeterministic decision tree complexity in Section
\ref{sec:dndtc}.  We then prove that $\Omega(\sqrt{N})$ oracle queries
are required to compute any $N$ bit Boolean function with
nondeterministic decision tree complexity $N$ in Section
\ref{sec:b4r}.  Finally in Section \ref{sec:lqbcbf} we present our
most general result: $\Omega(\sqrt[4]{D(f)})$ oracle queries are
required to compute a class of Boolean functions that meet a
sensitivity condition and have deterministic decision tree complexity
$D(f)$.

\section{Tree Functions}
\label{sec:tree}

Tree functions are Boolean functions of $N$ bits that can be written
as a disjunction of conjunctions in which each of the $N$ variables
occurs exactly once.  Tree functions are evasive in the classical case
\cite{lovasz94evasive}.

\begin{theorem}
\label{theo:treelow}
$\Omega(\sqrt[4]{N})$ oracle queries are required to compute any tree
function of $N$ bits in the bounded error setting.
\end{theorem}

\begin{proof}
For any tree function $f$ let $d$ be the number of terms, and let
$c_{max}$ be the maximum number of variables in any conjunction.

The tree function $f$ has a conjunctive term of size $c_{max}$.  Let
$x$ be the input that gives every variable in that term the value 1,
and every other variable 0, so that $f(x) = 1$.  The inputs attained
by negating one of the $c_{max}$ 1's in $x$ will yield $0$ when
evaluated by $f$.  Lemma \ref{lm:1xky} implies that
$\Omega(\sqrt{c_{max}})$ oracle queries are required to compute $f$.

The tree function $f$ also has $d$ terms.  Let $x$ be the input that
vices the first variable of every term the value 0 and every other
variable the value 1, so that $f(x) = 0$.  The inputs attained by
negating one of the $d$ 0's in $x$ yield $1$ when evaluated by $f$.
Lemma \ref{lm:1xky} now implies that $\Omega(\sqrt{d})$ oracle queries
are required to compute $f$.

Since $d \ge N/c_{max}$, either $c_{max} > \sqrt{N}$ or $d \ge
\sqrt{N}$, and the theorem follows.
\end{proof}

This lower bound is a special case of the $\Omega(\sqrt[4]{D(f)})$
lower bound for monotone functions proved by Beals et al.\
\cite{beals98quantum}, as tree functions are monotone functions with
decision tree complexity $N$.  Here we see for the first time a gap
that is not quadratic with the classical case.  While this lower bound
agrees with the result for monotone functions, the author believes it
is not asymptotically tight.  Whether there is an $\Omega(\sqrt{N})$
lower bound for computing $N$-bit tree functions is an open question.

\section{Nondeterministic Decision Tree Complexity}
\label{sec:dndtc}

In the presentation of the most general results of this thesis we
refer frequently to deterministic and nondeterministic decision tree
complexity, deterministic decision trees were discussed in Section
\ref{subs:ddt}.  Readers familiar with these topics can proceed
directly to Section \ref{sec:b4r}.

Let $f$ be an $N$-bit Boolean function.  We will use the following
definitions of L{\'{a}}szl{\'{o}} Lov{\'{a}}sz and P{\'{e}}ter
G{\'{a}}cs \cite{lovasz94complexity}.

For every input $x$ let $D(f,x)$ denote the minimum number of bits of
$x$ we could be told to convince us that $f(x)$ takes on a particular
value.  For example: $D($OR$,0^{N}) = N$.  If we are told fewer than
$N$ bits of an input are all 0 we can not determine the value of OR
for that input.  For any other input $x \neq 0^{N}$, $D($OR$,x) = 1$,
because we only need to be told that one bit of the input is a 1.  We
are not concerned with how many oracle queries we have to make in the
worst case to determine $f(x)$, but rather how many bits we could be
told in the best case to be sure of the value $f$ takes on $x$.

\begin{defi}
\label{defi:nd0}
$D_{0}(f) = \max\{D(f,x) | f(x) = 0\}$.  $D_{0}(f)$ is the
nondeterministic decision tree complexity of verifying that an input
takes on $0$ when evaluated by $f$.
\end{defi}

\begin{defi}
\label{defi:nd1}
$D_{1}(f) = \max\{D(f,x) | f(x) = 1\}$.  $D_{1}(f)$ is the
nondeterministic decision tree complexity of verifying that an input
takes on $1$ when evaluated by $f$.
\end{defi}

\begin{defi}
\label{defi:ndc}
$N(f) = \max\{D_{0}(f),D_{1}(f)\} = \max_{x}\{D(f,x)\}$.  The
nondeterministic decision tree complexity of computing an $N$-bit
Boolean function $f$ is the maximum of the nondeterministic decision
tree complexities of verifying any input of $f$ takes on one of
$\{0,1\}$.
\end{defi}

\section{Nondeterministically Evasive Functions}
\label{sec:b4r}

With an understanding of decision tree complexity we can now prove a
lower bound on a class of evasive functions.  In analogy to evasive
functions whose deterministic decision tree complexity is $N$ we call
functions with nondeterministic decision tree complexity $N$
\emph{Nondeterministically evasive}.  Every nondeterministically evasive
functions is evasive.

\begin{theorem}
\label{th:b2r}
$\Omega(\sqrt{N})$ oracle queries are required to compute any
nondeterministically evasive $N$-bit Boolean function in the bounded
error setting.
\end{theorem}

\begin{proof}
We will prove that any nondeterministically evasive $N$-bit Boolean
function $f$ has an input $x$ such that for at least half of the
Hamming neighbors of $x$ disagree with $x$ when evaluated by $f$.
Once this is proved the theorem will follow from Lemma \ref{lm:1xky}.

Assume to the contrary that for all inputs $x$ more than half of $x$'s
Hamming neighbors agree with $x$.  Consider any deterministic decision
tree that computes $f$.  Since $D(f) \ge N(f) = N$, there is some path
$P$ of length $N$ in the tree.  Call the variables along this path $a,
b, \ldots , z$, where each label corresponds to a unique number
between 1 and N inclusive.  Let $x_{a}, x_{b}, \ldots , x_{z}$ be the
values of the corresponding bits along the path.  Let us rearrange the
order of the input bits so that $a$ is the first input bit, $b$ is the
second input bit, and $z$ is the $N$th input bit.  Then path $P$ tells
us $f(x_{a}x_{b}\ldots x_{z})$ is either $0$ or $1$, and
$f(x_{a}x_{b}\ldots\overline{x_{z}})$ is the other.  Without loss of
generality let $f(x_{a}x_{b}\ldots x_{z}) = 1$ and
$f(x_{a}x_{b}\ldots\overline{x_{z}}) = 0$.

By the pigeonhole principle, there is some bit $i < N-1$ such that
\[
\begin{array}{ll}
	f(x_{a}x_{b}\ldots x_{i}\ldots x_{z}) = 1 & 	f(x_{a}x_{b}\ldots x_{i} \ldots \overline{x_{z}}) = 0 \\
	f(x_{a}x_{b}\ldots\overline{x_{i}}\ldots x_{z}) = 1 & 	f(x_{a}x_{b}\ldots\overline{x_{i}}\ldots\overline{x_{z}}) = 0 .
\end{array}
\]
However, if this is the case then we do not need to ask about the
$i$th bit on path $P$.  This argument follows for any path of length
$N$ in our deterministic decision tree.  In this case $D_{0}(f)$ and
$D_{1}(f)$ are at most $N-1$.  $N(f)$ is then $N-1$, a contradiction
to the condition that $N(f) = N$.  Therefore our assumption was false,
and there is at least one input $x$ such that at least half of $x$'s
Hamming neighbors yield a different value than $x$ when evaluated by
$f$.  By Lemma \ref{lm:1xky} $\Omega(\sqrt{N})$ oracle queries are
required to compute $f$ in the bounded error setting.
\end{proof}

This proof establishes that nondeterministically evasive functions
have inputs that are sensitive to negation on at least half of their
bits; the result then follows from Lemma \ref{lm:1xky}.  Not all
evasive functions are nondeterministically evasive.  OR is an example
of a nondeterministically evasive function; since Beals et al.\
provide an $O(\sqrt{N})$ algorithm to compute the OR of $N$ bits, this
lower bound is asymptotically tight.

\section{Sensitive Functions}
\label{sec:lqbcbf}

Nearly all the functions discussed so far have been evasive (or
conjectured to be so).  In our final result we consider functions that
are not necessarily evasive.  We call a Boolean function $f$
\emph{sensitive} if for some input $x$, there are $\Omega(N(f))$ Hamming
neighbors $y$ of $x$ such that $f(x) \neq f(y)$.  It is an open
question whether there are any nonsensitive functions.

\begin{theorem}
\label{th:b4r}
$O(\sqrt[4]{D(f)})$ oracle queries are required to compute any
sensitive function $f$ in the bounded error setting.
\end{theorem}

\begin{proof}
By Lemma \ref{lm:1xky}, $\Omega(\sqrt{N(f)})$ oracle queries are
required to compute any sensitive Boolean function. 

Lov{\'{a}}sz and G{\'{a}}cs \cite{lovasz94complexity} proved that
$D(f) \le D_{0}(f) \cdot D_{1}(f)$, and by definition, $N(f) =
\max\{D_{0}(f),D_{1}(f)\}$.  Thus, $N(f) = \Omega(\sqrt{D(f)})$, 
and the theorem follows.
\end{proof}

Beals et al.\ proved $\Omega(\sqrt[6]{D(f)})$ oracle
queries are required to compute any Boolean function in the bounded
error setting, but they suspect that this lower bound is not tight
\cite{beals98quantum}.  An open question is whether there are any 
functions which are not sensitive.  If not, then
$\Omega(\sqrt[4]{D(f)})$ oracle queries are required to compute any
Boolean function, which would be an improvement over what is currently
known.

